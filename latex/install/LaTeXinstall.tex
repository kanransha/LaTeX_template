\documentclass[a4paper,10pt,twocolumn,fleqn]{jarticle}

% call your preamble to get the package
\usepackage{common_preamble}


\begin{document}

\section{Tex}
堀・藤本研究室では、Tex(読み方は、テフ、テックなど)という文章組版システムを用い
て論文・資料作成を行います。組版とは、文章などのレイアウトを行うなどという意味です。
Tex は Microsoft Word や一太郎などのワープロソフトとは違い、ソースファイルを作成し、
コンパイル、プレビューという手順を踏みます。
その他の tex の特徴として以下のものが挙げられます。

\begin{itemize}
	\item  数式の出力が非常に綺麗で、高品質な組版が可能であること
	\item 現存する様々な OS に移植されていること
\end{itemize}	

\subsection{Latexとは}
T E X にはマクロ機能があり、 L A T E X とはこのマクロ機能を使って T E X をより使いやすくした
ものです。文書の体裁がある程度決まっている場合、マクロをそれ用に組んでおけば簡単に T E X
の文章を作成することができます。 ( 例えばコンパイル簡単にできる ) そのマクロの中でも特に
よく使われるものが LA T E X です。このマニュアルも L A T E X を用いて作ってあります。
学会で論文発表する場合には、 T E X を使って論文を作成するのは必須です。必ず覚えて下
さい。

\subsection{関連ファイルと基本的な内容}
*この内容は曖昧な記憶にもとづいています,大きな間違いは無いはずですが詳細な調査に基づく簡素な記述による更新が待たれます。
\subsubsection*{.texファイル}
文章の本体,本文が記述されているファイル。
このファイルをplatexコマンドによりコンパイルすることでPDFの前身であるdviファイルが作成されます。
\subsubsection*{.styファイル}
スタイルファイルと呼ぶことが多い。パッケージや文書のレイアウト等の記述を行う。
C言語でいうところのincludeするヘッダみたいなものか。
\subsubsection*{.auxファイル}
tex中のref等の対応を記録しているファイル。
\subsubsection*{.dviファイル}
texをコンパイルして出来るファイル。
ここからdvip系のコマンドをうつことでpdfに変換することが出来る。
\subsubsection*{.bibファイル}
参考文献のリストの情報を保存したファイル。
これを用いることでbibriographyをいちいち書かずに済む,
作成にはMendeley等の文献管理ソフトを併用するのが比較的便利。

\subsection{pLatexを用いたコンパイルの基本的な流れ}
後に説明するパッケージをインストールした後のコンパイルの流れは以下の通りです。
\begin{enumerate}
\item platexコマンドでtexファイルをdviに変換します。
この際,refや参考文献等の情報をもったauxファイルも同時に作成されます。

\item この後,dvipdfmxコマンドを用いることでこのdviファイルをpdfへと変換します。
この過程では図表を貼り付ける
\end{enumerate}
一般的には統合開発環境でこれらを制御するので直接これらのコマンドを触ることはあまりありませんが,覚えておいて損はないです。

\section{LatexのInstall}
Latexを制御するには様々なパッケージやソフトをインストールする必要がありますが,
windowsの日本語版であれば有志がこれを簡単にインストールすることのできるTexインストーラを作成してくれています。

以下のWebサイトを参照。
研究室のwikiも。


\section{推奨エディタ:TexStudio}
上記のTexインストーラを用いることでデフォルトでtexworksというエディタが付属します。
これも使い勝手は悪くないのですが,私個人としてはTexStudioをおすすめします。

これは様々なマクロが用意してある点と後述の幾つかのアプリケーションとの連携が比較的容易なため,おすすめしているエディタになります。


\subsection{SumatraPDF}
デフォルトで入っているAdobeReaderでPDFを開くかと思いますが,殊Latexで文章を書く場合に至っては
SumatraPDFというフリーソフトをおすすめします。

このソフトのAdobeに対する利点は主に以下の2点です。
\begin{itemize}
	\item 開いているPDFファイルに対する変更を随時確認し,反映する。
	\item PDFをダブルクリックすることによるTex文書の逆探索が可能である。
\end{itemize}
前者では,「xxx.tex」というTexをコンパイルして「xxx.pdf」というPDFファイルを作成する際,
Adobe等によってPDFファイル「xxx.pdf」を開いているとこれを更新できないという問題があります。
SumatraPDFを使うことによりこの問題を回避することが出来ます。

後者の機能は「inverse search」と呼ばれ,例えば「xxx.pdf」の変更したい部分をダブルクリックすることで該当するTexファイル「xxx.tex」の該当位置を開くことが出来るという機能で文章の校正に非常に便利です。

これらの機能はTexStudioとの連携を通して確認することが出来ます。


\section{}








\end{document}
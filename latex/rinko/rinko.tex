\documentclass[a4paper,fleqn,twocolumn,9pt]{jsarticle}
\title{XXXXXXXXXに関する調査\\
Survey on HFlab}
\author{D1 37-XXXXXX Horifuji Taro}

% 電気系輪講用パッケージ
\usepackage{dendentitle}

%英字フォント
\usepackage{times}
\usepackage{booktabs}
% 図の挿入
\usepackage[dvipdfmx]{graphicx}
\usepackage{subfigure}
% 数式
\usepackage{amsmath}
\usepackage{ascmac}
\usepackage{pifont}
% 余白調整
\usepackage{geometry}
\geometry{left=25mm,right=25mm,top=30mm,bottom=30mm}


%その他諸々 コメントアウト可
\usepackage{listings}
\usepackage{url}
\usepackage{multirow}
\usepackage{latexsym}
\usepackage{epsf}
%\usepackage{docmute} %単体でコンパイルできるようにする
\usepackage{mathtools} % dcases を使う

% ハイパーリンク用
\usepackage[dvipdfmx, bookmarkstype=toc, colorlinks=false, pdfborder={0 0 0}, bookmarks=true, bookmarksnumbered=true]{hyperref}
\usepackage{pxjahyper}


%%%%%%% 便利マクロ  %%%%%%%%%%%%%%%%%%%%%%%%%%
\newcommand{\bm}[1]{\mbox{\boldmath $#1$}}
\newcommand{\eq}[1]{Eq.(\ref{eq:#1})}
\newcommand{\zu}[1]{図\ref{fig:#1}}
\newcommand{\shiki}[1]{式(\ref{eq:#1})}
\newcommand{\fig}[1]{Fig.\ \ref{fig:#1}}
\newcommand{\grp}[1]{Fig.\ \ref{grp:#1}}
\newcommand{\tab}[1]{Table\ \ref{tab:#1}}
\newcommand{\defeq}{:=}
\newcommand{\bvec}[1]{\mbox{\boldmath $#1$}} %太字
\newcommand{\tvec}[1]{\bvec{#1}^{\mathsf{T}}} %太字転置
%%%%%%%%%%%%%%%%%%%%%%%%%%%%%%%%%%%%%%%%%

% 図のPath
\graphicspath{{./fig/}}

\begin{document}
\makedendentitle{D1 Survey}{Hori-Fujimoto Lab\\Department of Electrical Engineering }


%
%
%%%%%%%%%%%%%%%%%%%%%%%%%%%%%%%%%%%%%%%%%%%%%%%%%%%%%%%%%%%%%%%%%%%%%%%%%%%%%%
%\begin{abstract}
%\end
%%%%%%%%%%%%%%%%%%%%%%%%%%%%%%%%%%%%%%%%%%%%%%%%%%%%%%%%%%%%%%%%%%%%%%%%%%%%%%%
%%%%%%%%%%%%%%%%%%%%%%%%%%%%%%%%%%%%%%%%%%%%%%%%%%%%%%%%%%%%%%%%%%%%%%%%%

%\begin{jkeyword}
%装着型ロボット、外骨格ロボット、リハビリテーション
%\end{jkeyword}
%
%\begin{ekeyword}
%Wearable robot, Exoskeleton、Rehabilitation
%\end{ekeyword}
%%%%%%%%%%%%%%%%%%%%%%%%%%%%%%%%%%%%%%%%%%%%%%%%%%%%%%%%%%%%%%%%%%%%%%%%%%%%%%%
%\maketitle
\section{序論}
序論だよ。


\section{まとめ}


{\small
\bibliographystyle{ieeetr}
\bibliography{ref.bib}
}
%IEEEの自動化農業機械のHP。
%\url{http://www.ieee-ras.org/agricultural-robotics}
%小曽根淳、『紅毛流測量術の起源とその教育的、科学史的意義について』、2011年度科学研究費症例研究。研究課題番号:23909014

\end{document}

\chapter*{発表文献}
\section*{掲載済みの論文誌}
\noindent
\begin{tabular}{ccl}
[1]&著\hspace{2em}者&\underline{大西 亘},藤本博志,堀 洋一,坂田晃一,鈴木一弘,佐伯和明\\
&題\hspace{2em}名&超精密ステージにおけるオイラーの運動方程式と物体座標系の回転の\\
 & & 非線形性と軸間干渉を補償した姿勢制御法\\
&論文誌名\hspace{0.5em}&電気学会論文誌D, vol. 134--D, no. 3, pp. 293--300 (2014)\\
\end{tabular}\\

\section*{投稿準備中の論文誌}
\noindent
\begin{tabular}{ccl}
[2]&著\hspace{2em}者&\underline{W. Ohnishi}, H. Fujimoto, K. Sakata, K. Suzuki, and K. Saiki\\
&題\hspace{2em}名&Decoupling Control Method for High-Precision Stages \\
&& using Multiple Actuators considering the Misalignment among \\
 & & the Actuation Point, Center of Gravity, and Center of Rotation\\
&論文誌名\hspace{0.5em}& IEEJ Journal of Industry Applications (to be submitted)\\
\end{tabular}\\
\\
\\
\begin{tabular}{ccl}
[3]&著\hspace{2em}者&\underline{W. Ohnishi}, H. Fujimoto, K. Sakata, K. Suzuki, and K. Saiki\\
&題\hspace{2em}名&General Two Rigid Body Model with Rotational Constraint and \\
&&Integrated Design of Mechanism and Control\\
&論文誌名\hspace{0.5em}& IEEE Transaction on Industrial Electronics (to be submitted)\\
\end{tabular}\\


\section*{査読付国際会議論文}
\noindent
\begin{tabular}{ccl}
[4]&著\hspace{2em}者&\underline{W. Ohnishi}, H. Fujimoto, K. Sakata, K. Suzuki, and K. Saiki\\
&題\hspace{2em}名&Proposal of Attitude Control for High-Precision Stage by Compensating\\
&& Nonlinearity and Coupling of Euler's Equation and Rotational Kinematics\\
&会\hspace{0.5em}議\hspace{0.5em}名& 39th Annual Conference of the IEEE Industrial Electronics Society  \\
&場\hspace{2em}所&Vienna, Austria\\
&発\hspace{0.5em}表\hspace{0.5em}日& 13th, November, 2013\\
&開催期間\hspace{0.5em}& 10th--13th, November, 2013 \\
\end{tabular}\\
\\
\\
\begin{tabular}{ccl}
[5]&著\hspace{2em}者&\underline{W. Ohnishi}, H. Fujimoto, K. Sakata, K. Suzuki, and K. Saiki\\
&題\hspace{2em}名&Design and Control of 6-DOF High-Precision Scan Stage with Gravity Canceller\\
&会\hspace{0.5em}議\hspace{0.5em}名& 2014 American Control Conference \\
&場\hspace{2em}所&Portland, USA\\
&発\hspace{0.5em}表\hspace{0.5em}日& 4th, June, 2014\\
&開催期間\hspace{0.5em}& 4th--6th, June 2014 \\
\end{tabular}\\
\\
\\
\begin{tabular}{ccl}
[6]&著\hspace{2em}者&\underline{W. Ohnishi}, H. Fujimoto, K. Sakata, K. Suzuki, and K. Saiki\\
&題\hspace{2em}名&Proposal of Decoupling Control Method for High-Precision Stages \\
&& using Multiple Actuators considering the Misalignment among \\
 & & the Actuation Point, Center of Gravity, and Center of Rotation\\
&会\hspace{0.5em}議\hspace{0.5em}名& The 1st IEEJ International Workshop on Sensing, Actuation, and Motion Control \\
&場\hspace{2em}所&Nagoya, Japan\\
&発\hspace{0.5em}表\hspace{0.5em}日& March, 2015 (accepted)\\
&開催期間\hspace{0.5em}& 9th--10th, March 2015 \\
\end{tabular}\\
\\
\\
\begin{tabular}{ccl}
[7]&著\hspace{2em}者&\underline{W. Ohnishi}, H. Fujimoto, K. Sakata, K. Suzuki, and K. Saiki\\
&題\hspace{2em}名&Integrated Design of Mechanism and Control for High-Precision Stage\\
&&by the Interaction Index in the Direct Nyquist Array Method\\
&会\hspace{0.5em}議\hspace{0.5em}名&  2015 American Control Conference \\
&場\hspace{2em}所&Chicago, USA\\
&発\hspace{0.5em}表\hspace{0.5em}日& July, 2015 (accepted)\\
&開催期間\hspace{0.5em}& 1st--3rd, July, 2015 \\
\end{tabular}\\


\section*{国内会議論文}
\noindent
%
\begin{tabular}{ccl}
[8]&著\hspace{2em}者&\underline{大西 亘},藤本博志,堀 洋一,坂田晃一,鈴木一弘,佐伯和明\\
&題\hspace{2em}名&超精密ステージにおけるオイラーの運動方程式の非線形性と\\
&&軸間干渉を補償した姿勢制御の一提案\\
&会\hspace{0.5em}議\hspace{0.5em}名&平成25年産業計測制御/メカトロニクス制御合同研究会, \\
&&IIC-13-101, MEC-13-101, pp.67-72, 2013\\
&場\hspace{2em}所&千葉大学,千葉県\\
&発\hspace{0.5em}表\hspace{0.5em}日& 2013年3月8日\\
&開催期間\hspace{0.5em}& 2013年3月7--8日 \\
\end{tabular}\\
\\
\\
\begin{tabular}{ccl}
[9]&著\hspace{2em}者&\underline{大西 亘},藤本博志,坂田晃一,鈴木一弘,佐伯和明\\
&題\hspace{2em}名&超精密ステージにおける非線形性と軸間干渉を有する回転運動の干渉解析\\
&会\hspace{0.5em}議\hspace{0.5em}名&平成25年メカトロニクス制御研究会「ナノスケールサーボのための制御技術」, \\
&&MEC-13-169, pp. 61--66. 2013\\
&場\hspace{2em}所&東京電機大学,東京\\
&発\hspace{0.5em}表\hspace{0.5em}日& 2013年9月5日\\
&開催期間\hspace{0.5em}& 2013年9月5日 \\
\end{tabular}\\
\\
\\
\begin{tabular}{ccl}
[10]&著\hspace{2em}者&\underline{大西 亘},藤本博志,坂田晃一,鈴木一弘,佐伯和明\\
&題\hspace{2em}名&ナイキスト配列法における干渉指数を用いた\\
&&超精密ステージの機構と制御の統合設計の提案\\
&会\hspace{0.5em}議\hspace{0.5em}名&平成26年メカトロニクス制御研究会「ナノスケールサーボのための制御技術」, \\
&&MEC-14-148, pp. 1--6. 2014\\
&場\hspace{2em}所&電気学会会議室,東京\\
&発\hspace{0.5em}表\hspace{0.5em}日& 2014年9月1日\\
&開催期間\hspace{0.5em}& 2014年9月1日 \\
\end{tabular}\\



\section*{共著査読付国際会議論文}
\noindent
\begin{tabular}{ccl}
[11]&著\hspace{2em}者&Binh Minh Nguyen, Kiyoto Ito, \underline{W. Ohnishi}, Yafei Wang, Hiroshi Fujimoto, Yoichi Hori, \\
&&Masaki Odai, Hironori Ogawa, Erii Takano, Tomohiro Inoue, and Masahiro Koyama\\
&題\hspace{2em}名&Dual Rate Kalman Filter Considering Delayed Measurement \\
&&and Its Application in Visual Servo \\
&会\hspace{0.5em}議\hspace{0.5em}名& The 13th International Workshop on Advanced Motion Control \\
&場\hspace{2em}所&Yokohama, Japan\\
&発\hspace{0.5em}表\hspace{0.5em}日& 15th, March, 2014\\
&開催期間\hspace{0.5em}& 14th--16th, March, 2014 \\
\end{tabular}\\
\\
\\
\begin{tabular}{ccl}
[12]&著\hspace{2em}者&Riccardo Antonello, Roberto Oboe, Stefano Bizzotto, Emanuele Siego, \\
&&Yuma Yazaki, \underline{W. Ohnishi}, and Hiroshi Fujimoto\\
&題\hspace{2em}名&Feasible trajectory generation for a dual stage positioning system \\
&&using a simplified model predictive control approach \\
&会\hspace{0.5em}議\hspace{0.5em}名& IEEE/IES International Conference on Mechatronics \\
&場\hspace{2em}所&Nagoya, Japan\\
&発\hspace{0.5em}表\hspace{0.5em}日& March, 2015 (to be presented)\\
&開催期間\hspace{0.5em}& 6th--8th, March, 2015 \\
\end{tabular}\\

%

\section*{受\hspace{.5zw}賞}
\noindent
\begin{tabular}{ccl}
[13]&受\hspace{0.5em}賞\hspace{0.5em}者&\underline{大西 亘}\\
&題\hspace{2em}名&超精密ステージにおけるオイラーの運動方程式の\\
&&非線形性と軸間干渉を補償した姿勢制御に関する研究 \\
&受\hspace{0.5em}賞\hspace{0.5em}名& 東京大学工学部 電子情報工学科・電気電子工学科 平成24年度 学科長特別賞 \\
&受\hspace{0.5em}賞\hspace{0.5em}日&2013年3月26日\\
\end{tabular}\\
\\
\\
\begin{tabular}{ccl}
[14]&受\hspace{0.5em}賞\hspace{0.5em}者&\underline{大西 亘}\\
&受\hspace{0.5em}賞\hspace{0.5em}名& 電気学会東京支部 電気学術奨励賞 \\
&受\hspace{0.5em}賞\hspace{0.5em}日&2013年3月31日\\
\end{tabular}\\
\\
\\
\begin{tabular}{ccl}
[15]&受\hspace{0.5em}賞\hspace{0.5em}者&\underline{大西 亘}\\
&題\hspace{2em}名&超精密ステージにおける非線形性と軸間干渉を有する回転運動の干渉解析\\
&受\hspace{0.5em}賞\hspace{0.5em}名& 電気学会 メカトロニクス技術委員会優秀論文賞 \\
&受\hspace{0.5em}賞\hspace{0.5em}日&2014年1月10日\\
\end{tabular}\\
\\
\\
\begin{tabular}{ccl}
[16]&受\hspace{0.5em}賞\hspace{0.5em}者&\underline{大西 亘}\\
&題\hspace{2em}名&超精密ステージにおける非線形性と軸間干渉を有する回転運動の干渉解析\\
&受\hspace{0.5em}賞\hspace{0.5em}名& 電気学会産業応用部門 部門優秀論文賞 \\
&受\hspace{0.5em}賞\hspace{0.5em}日&2014年3月31日\\
\end{tabular}\\
\\
\\
\begin{tabular}{ccl}
[17]&受\hspace{0.5em}賞\hspace{0.5em}者&\underline{大西 亘}\\
&題\hspace{2em}名&6自由度超精密位置決めステージの非干渉化制御に関する研究\\
&受\hspace{0.5em}賞\hspace{0.5em}名& 公益財団法人 電気電子情報学術振興財団 原島博学術奨励賞 \\
&受\hspace{0.5em}賞\hspace{0.5em}日&2014年5月28日\\
\end{tabular}\\
\\
\\
\begin{tabular}{ccl}
[18]&受\hspace{0.5em}賞\hspace{0.5em}者&\underline{大西 亘}\\
&題\hspace{2em}名&超精密ステージにおける非線形性と軸間干渉を有する回転運動の干渉解析\\
&受\hspace{0.5em}賞\hspace{0.5em}名& 電気学会 優秀論文発表賞A賞 \\
&受\hspace{0.5em}賞\hspace{0.5em}日&2014年9月1日\\
\end{tabular}\\
\\
\\
\begin{tabular}{ccl}
[19]&受\hspace{0.5em}賞\hspace{0.5em}者&\underline{大西 亘}\\
&題\hspace{2em}名&ナイキスト配列法における干渉指数を用いた\\
&&超精密ステージの機構と制御の統合設計の提案\\
&受\hspace{0.5em}賞\hspace{0.5em}名& 電気学会 メカトロニクス技術委員会優秀論文賞 \\
&受\hspace{0.5em}賞\hspace{0.5em}日&2015年1月7日\\
\end{tabular}\\

%% v3.0 [2011/03/08]
%% 1. 和文「論文」(「解説」を含む)
%% 2. 和文「研究開発レター」
%% 3. 英文「論文/研究開発レター」
%% 4. 英文「Extended Summary」
%%   の各タイプ別のテンプレートです

%% 1. 和文「論文」用テンプレート
\documentclass[usejistfm]{ieej}
%\usepackage[fleqn]{amsmath}
\usepackage{../common_preamble/common_preamble} 


\FIELD{}
\YEAR{2016}
\NO{1}
\jtitle{電気学会論文誌用フォーマット}
%\jtitle[]{}
\etitle{}
\authorlist{%
 \authorentry{ホリフジ タロウ}{HF MAN}{s}{UTe}
 \authorentry{藤本 博志}{Hiroshi Fujimoto}{S}{UTe}
% \authorentry{}{}{}{}
}
\affiliate[UTe]{東京大学工学系研究科 \\〒277-8561 千葉県柏市柏の葉5-1-5 }{The University of Tokyo \\ 277-8561, 5-1-5, Kashiwanoha, Kashiwa, Chiba, Japan}

%\affiliate[]{ \\ }{ \\ }
%\affiliate[]{ \\ }{ \\ }

%\received{}{}{}
%\revised{}{}{}

\begin{document}
\begin{abstract}
abstract
\end{abstract}
\begin{jkeyword}
キーワード
\end{jkeyword}
\begin{ekeyword}
keyword
\end{ekeyword}
\maketitle

\section{はじめに}
電気学会論文誌用のフォーマットです。

\begin{thebibliography}{99}% 文献が10以上のとき99,10未満のとき9
\bibitem{}
\bibitem{}
\end{thebibliography}

\appendix
\section{}

\begin{biography}
\profile{}{}{}
%\profile{}{}{}
%\profile*{}{}{}
\end{biography}
\end{document}


%% 2. 和文「研究開発レター」用テンプレート
\documentclass[usejistfm,letter]{ieej}
\usepackage{graphicx}
%\usepackage[fleqn]{amsmath}
\usepackage[varg]{txfonts}

\FIELD{}
\YEAR{2016}
\NO{1}
\jtitle{}
%\jtitle[]{}
\etitle{}
\authorlist{%
 \authorentry{}{}{}{}
% \authorentry{}{}{}{}
% \authorentry{}{}{}{}
}
\affiliate[]{ \\ }{ \\ }
%\affiliate[]{ \\ }{ \\ }
%\affiliate[]{ \\ }{ \\ }

%\received{}{}{}
%\revised{}{}{}

\begin{document}
\begin{abstract}

\end{abstract}
\begin{jkeyword}

\end{jkeyword}
\begin{ekeyword}

\end{ekeyword}
\maketitle

\section{}


\begin{thebibliography}{99}% 文献が10以上のとき99,10未満のとき9
\bibitem{}
\bibitem{}
\end{thebibliography}

\appendix
\section{}

\begin{biography}
\profile{}{}{}
%\profile{}{}{}
%\profile*{}{}{}
\end{biography}
\end{document}


%% 3. 英文「論文/研究開発レター」用テンプレート
\documentclass[english]{ieej}
%\documentclass[english,letter]{ieej}
\usepackage{graphicx}
%\usepackage[fleqn]{amsmath}
\usepackage[varg]{txfonts}

\FIELD{}
\YEAR{2011}
\NO{1}
\title[]{}
\authorlist{%
 \authorentry{}{}{}
% \authorentry{}{}{}
% \authorentry{}{}{}
}
\affiliate[]{ \\ }
%\affiliate[]{ \\ }
%\affiliate[]{ \\ }

%\received{}{}{}
%\revised{}{}{}

\begin{document}
\begin{abstract}

\end{abstract}
\begin{keyword}

\end{keyword}
\maketitle

\section{}


\begin{thebibliography}{99}% 文献が10以上のとき99,10未満のとき9
\bibitem{}
\bibitem{}
\end{thebibliography}

\appendix
\section{}

\begin{biography}
\profile{}{}{}
%\profile{}{}{}
%\profile{}{}{}
\end{biography}
\end{document}


%% 4. 英文「Extended Summary」用テンプレート
\documentclass[english,ExtendedSummary]{ieej}
\usepackage{graphicx}
%\usepackage{latexsym}
%\usepackage[fleqn]{amsmath}
\usepackage[varg]{txfonts}

\title{}
\authorlist{%
 \authorentry{}{}{}
% \authorentry{}{}{}
}
\affiliate[]{}
%\affiliate[]{}

\begin{document}
\begin{keyword}
% keyword list
\end{keyword}
\maketitle

\end{document}

\documentclass{hfthesis} % クラスファイルにまとめました
\usepackage{preamble} % 適宜プリアンブルをpreamble.styに追加
% 表紙
\affiliation{東京大学 工学部 電気電子工学科}
\thesistype{2018}{解説資料}
\title{クラスファイル ``hfthesis'' の使い方 \\(まだ未完成です。.texファイルをみて察してください)}
% 指導教員は2人まで指定可能
\professor{堀 洋一}{教授}
\professor{藤本 博志}{准教授}
\date{yyyy年mm月dd日提出}
\author{00-000000}{堀藤 太郎}

\begin{document}
    %タイトルページを作成
    \maketitle
    % 概要を追加
    \includeabstract{chapter/abstract}
    % 目次,図目次,表目次を追加
    \makemokuji

    % チャプタごとに別ファイルに
    \documentclass[a4paper,11pt,oneside,openany,fleqn]{jsbook}

\usepackage{preamble}

\begin{document}

    \chapter{Introduction}
        Write introduction here.

\end{document}

    \documentclass[12pt,a4paper,oneside,onecolumn,fleqn,dvipdfmx,report]{jsbook}

\usepackage{../main/preamble}

\begin{document}

    \chapter{Setup}
        Write setup here.

\end{document}

    \documentclass[12pt,a4paper,oneside,onecolumn,fleqn,dvipdfmx,report]{jsbook}

\usepackage{../main/preamble}

\begin{document}

    \chapter{Modeling}
        Write modeling here.

\end{document}


    \part{理論}

    \documentclass[fleqn]{jreport}

\usepackage{../main/preamble}

\begin{document}

    \chapter{従来法}
        ここに従来法を書く。

\end{document}

    \documentclass[a4paper,11pt,oneside,openany,fleqn]{jsbook}

\usepackage{../main/preamble}

\begin{document}

    \chapter{Proposed Method}
        Write proposed method here.

\end{document}


    \part{応用}

    \documentclass[12pt,a4paper,oneside,onecolumn,fleqn,dvipdfmx,report]{jsbook}

\usepackage{../main/preamble}

\begin{document}

    \chapter{Simulation}
        Write simulation here.

\end{document}

    \documentclass[a4paper,11pt,oneside,openany,fleqn]{jsbook}

\usepackage{preamble}

\begin{document}

    \chapter{Experiment}
        Write experiment here.

\end{document}

    \documentclass[a4paper,11pt,oneside,openany,fleqn]{jsbook}

\usepackage{preamble}

\begin{document}

    \chapter{Conclusion}
        Write conclusion here.

\end{document}


    % 謝辞,付録,発表文献は特殊なチャプタです
    \includeacknowledgments{chapter/acknowledgments}
    \includeappendix{chapter/appendix}

    % 参考文献をBibTeXで
    \bibliographystyle{IEEEtran}
    \bibliography{ref}

    \includepublication{chapter/publication}

    \printindex

\end{document}

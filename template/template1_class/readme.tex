\documentclass{hfthesis} % クラスファイルにまとめました
\usepackage{preamble} % 適宜プリアンブルをpreamble.styに追加
% 表紙
\affiliation{東京大学 工学部 電気電子工学科}
\thesistype{2018}{解説資料}
\title{クラスファイル ``hfthesis'' の使い方 \\(まだ未完成です。.texファイルをみて察してください)}
% 指導教員は2人まで指定可能
\professor{堀 洋一}{教授}
\professor{藤本 博志}{准教授}
\date{yyyy年mm月dd日提出}
\author{00-000000}{堀藤 太郎}

\begin{document}
    %タイトルページを作成
    \maketitle
    % 概要を追加
    \includeabstract{chapter/abstract}
    % 目次,図目次,表目次を追加
    \makemokuji

    % チャプタごとに別ファイルに
    \include{chapter/introduction}
    \include{chapter/setup}
    \include{chapter/modeling}

    \part{理論}

    \include{chapter/conventional}
    \include{chapter/proposed}

    \part{応用}

    \include{chapter/simulation}
    \include{chapter/experiment}
    \include{chapter/conclusion}

    % 謝辞,付録,発表文献は特殊なチャプタです
    \includeacknowledgments{chapter/acknowledgments}
    \includeappendix{chapter/appendix}

    % 参考文献をBibTeXで
    \bibliographystyle{IEEEtran}
    \bibliography{ref}

    \includepublication{chapter/publication}

    \printindex

\end{document}

%%%
%% v2.0 [2011/03/14]
\documentclass[english]{ieej-e-samcon}
%\documentclass[english,letter]{ieej-e}
\usepackage{graphicx}% [dvips] [pdftex]
\usepackage{latexsym}
%\usepackage[fleqn]{amsmath}
\usepackage[varg]{txfonts}

\def\ClassFile{\texttt{ieej-e.cls}}
\def\PS{\textsc{Post\-Script}}
\def\AmSLaTeX{\leavevmode\hbox{$\cal A\kern-.2em\lower.376ex
 \hbox{$\cal M$}\kern-.2em\cal S$-\LaTeX}}
\def\BibTeX{{\rm B\kern-.05em{\sc i\kern-.025em b}\kern-.08em
 T\kern-.1667em\lower.7ex\hbox{E}\kern-.125emX}}

\YEAR{2017}
\title[How to Use the \LaTeXe\ Class File]
      {How to Use the \LaTeXe\ Class File (\ClassFile)\\
       for the Transactions of the Institute of Electrical Engineers of Japan}
\authorlist{%
 \authorentry[TaroDenshi@iee.or.jp]{Taro Denshi}{m}{TRL}
 \authorentry{Hanako Denki}{n}{KEC}
}
\affiliate[TRL]
 {Technical Research Labs., Shin-nichi Electric Co., Ltd. \\
  7--2, Gobancho, Chiyoda-ku, Tokyo, Japan 102--0076}
\affiliate[KEC]
 {Technical Labs., Kagoshima Electron Corp.\\
  2--100, Daikan-cho, Kagoshima, Japan 890--0099}

\begin{document}
\begin{abstract}
IEE Japan provides a \LaTeXe\ class file, named \ClassFile, 
for the Transactions of the Institute of Electrical Engineers of Japan. 
This document describes how to use the class file, 
and also makes some remarks about typesetting a document by using \LaTeXe. 
The design is based on \LaTeXe.
\end{abstract}
\begin{keyword}
Class file, \LaTeXe
\end{keyword}
\maketitle

\section{Introduction}
\label{sec:intro}

This document describes how to handle the \ClassFile\  
for the Transactions of the Institute of Electrical Engineers of Japan.
Section~\ref{template} explains how to typeset according to the template. 
\texttt{template.tex} which is distributed 
with the \ClassFile\ can be used. 
Section~\ref{generalnote} describes a special feature of \ClassFile, 
which is different to the \texttt{article.cls} provided 
by the standard \LaTeXe\ and which points may be 
aware of on writing a paper and so on. 
Section~\ref{typesetting} is about typographic notes, which explains 
how to typeset, how to prevent typographic errors 
and how to handle long formulas. 
For information about printing on A4 paper and making pdf file, 
see Appendix (p.\pageref{sec:pdf}). 

\section{Template and How to Typeset the Paper and Letter}
\label{template}

\subsection{Type of the Paper}

Here is the template of the paper. 

\begin{verbatim}
\documentclass[english]{ieej-e}
\usepackage{graphicx}
\usepackage[varg]{txfonts}
\FIELD{A}
\YEAR{2011}
\NO{1}
\title[]{title}
\authorlist{%
 \authorentry{name}{membership}{label}
}
\affiliate[label]
 {affiliate\\ address}
\received{2011}{1}{23}
\revised{2011}{3}{15}
\begin{document}
\begin{abstract}
Summary
\end{abstract}
\begin{keyword}
Keywords
\end{keyword}
\maketitle
\section{Introduction}
 ...
\begin{thebibliography}{99}
\bibitem{}
\bibitem{}
 ...
\end{thebibliography}
\appendix
\section{}
 ...
\acknowledgment
 ...
\begin{biography}
\profile{membership}{name}{text}
\end{biography}
\end{document}
\end{verbatim}

\begin{itemize}
\item
The txfonts package should be loaded. 
\begin{verbatim}
\usepackage[varg]{txfonts}
\end{verbatim}
Note that if you require \texttt{amsmath} packages, 
it must be specified before \texttt{txfonts} package. 

\item
The \verb/\FIELD/ command is used for the footer. 
Its argument indicates the categories of IEEJ transactions 
(see Table 1 of ``Author's Guidelines for the Transactions of 
the Institute of Electrical Engineers of Japan''). 
For example, in the case of ``Fundamental and Materials'' 
\texttt{A} is specified as the argument of \verb/\FIELD/. 

\item
The \verb/\YEAR/ command is used for the footer. 
The year of publication is assigned as \verb/\YEAR{2011}/. 

\item
The \verb/\NO/ command is also used for the footer. 
The number of the issue is assigned as \verb/\NO{1}/. 

Those two commands are optional. 
If you do not know which issue your paper to be appear, 
they may be commented out or left blank.

\item
The title of a paper is assigned in \verb/\title/. 
You may use \verb/\\/ to start a new line in a long title. 

The argument of the \verb/\title/ command is used for more 
than just producing a title, it is also used to generate a running head. 

A shorter title should be specified within 8 words as follows. 

\noindent
\verb/\title[/%
 \textit{short title}\verb/]{/%
  \textit{title}\verb/}/

\item
The outputs of authors' names, memberships and marks of affiliates 
are automatically generated by using the \verb/\authorlist/ 
and \verb/\authorentry/ commands. 

The \verb/\authorentry/ command must be described as 
an argument of the \verb/\authorlist/ command. 

The \verb/\authorentry/ command has three arguments. 

\noindent
\verb/\authorentry{/\textit{name}%
 \verb/}{/\textit{membership}%
  \verb/}{/\textit{label}%
   \verb/}/

For example, they could be typesetted as follows: 
\begin{verbatim*}
\authorlist{%
\authorentry{Taro Denshi}{m}{TRL}
\authorentry{Hanako Denki}{n}{KEC}
}
\end{verbatim*}

\begin{itemize}
\item
The first argument of \verb/\authorentry/ is filled 
with the author's name. 

\item
The second argument is specified by one letter 
out of seven letters (m, a, s, l, n, h, S, f), 
each one indicating the membership of authors 
as the following table shows. 

\onelineskip

\begin{center}
\begin{small}
\begin{tabular}{lll}
\hline
\texttt{m} & Member\rule{0mm}{3.5mm}\\
\texttt{a} & Associate\\
\texttt{s} & Student Member\\
\texttt{l} & Life Member\\
\texttt{n} & Non-member\\
\texttt{h} & Honorary Member\\
\texttt{S} & Senior Member\\
\texttt{f} & Fellow\\
\hline
\multicolumn{3}{p{65mm}}{\footnotesize 
 the left column is letters to be specified. 
 the right column is membership to be generated.}
\end{tabular}
\end{small}
\end{center}

\onelineskip

No extra spaces may be added between a letter and a brace. 
\verb*/{m}/ and \verb*/{m }/ are regarded as different. 
The latter will not generate ``Member''. 

\item
The third argument is assigned by the label of the author's affiliate, 
corresponding to the label of the \verb/\affiliate/ command (see below). 
For example, an abbreviation for a university, institute or company 
can be given. 

If an author does not have an affiliate, 
\texttt{none} must be specified. 

\item
You might specify e-mail address as follows. 
\begin{verbatim}
\authorentry[TaroDenshi@iee.or.jp]
 {Taro Denshi}{m}{TRL}
\end{verbatim}
The following words ared generated on the left side at the bottom 
of the first page.\hfil\break
``a) Correspondence to: TaroDenshi@iee.or.jp''

\item
The \verb/\authorentry/ command lists the authors' names and affiliates 
one by one. 
If more than 10 authors are listed, 
the list occupies almost a half of the page. 
Therefore, if you specify as follows, 
\begin{verbatim}
\def\authoralign{2}
\end{verbatim}
then, it will list the names two per line. 
The argument of \verb/\authoralign/ must be 1 or 2. 
\end{itemize}

%\item
%If you need to inform a pesent affiliate, 
%4th argument of the \verb/\authorentry/ command can be used 
%as follows. 
%
%\noindent
%\verb/\authorentry{/\textit{name}%
% \verb/}{/\textit{membership}%
%  \verb/}{/\textit{label}%
%   \verb/}[/
%    \textit{LABEL}%
%     \verb/]/
%
%The 4th argument is corresponding to 
%the label of a \verb/\paffiliate/ command (see below). 

\item
An author's affiliate is described 
in the \verb/\affiliate/ command as follows. 

\verb/\affiliate[/\textit{label}%
 \verb/]{/\textit{affiliate}%%
  \verb/\\/\textit{address}%%
  \verb/}/

The first argument ``\textit{label}'' must be the same as 
the 3rd argument of the \verb/\authorentry/ command. 
The second argument is assigned by both the author's affiliate and address, 
which are separeted by \verb/\\/. 

No extra spaces may be added between a letter and a brace 
in the first \textit{label} argument. 
The entry of \verb/\affiliate/ must be followed by the order 
of labels in \verb/\authorlist/ commands. 

%\item
%The author's present affiliate is described 
%in the \verb/\paffiliate/ command as follows. 
%
%\verb/\paffiliate[/\textit{label}%
% \verb/]{/\textit{affiliate}%%
%  \verb/\\/\textit{address}%%
%  \verb/}/
%
%The first argument \textit{label} must be the same as 
%the 4th argument of \verb/\authorentry/ command. 

\item
If the labels of \verb/affiliate/ are different 
from those of \verb/\authorentry/, 
there will be a warning message on your terminal.  

\item
Both the \verb/\received/ and \verb/\revised/ commands are 
used for the date of receipt and revision of the paper. 
For example, 
the date of receipt is assigned as \verb/\received{2011}{1}{23}/ and 
the date of revision is assigned as \verb/\revised{2011}{3}{15}/. 

Both commands will generate the following strings 
below authors' names. 

\noindent
``(Manuscript received January 23, 2011, revised March 15, 2011)''

\item
The text of the abstract is described 
in the \texttt{abstract} environment. 
The text should be a maximum of 150 to 200 words.

The text of the keywords is described 
in the \texttt{keyword} environment. 
The text should be a maximum of 6 words.

\item
The \verb/\maketitle/ command must come after those commands 
before the main text begins. 

\item
The \verb/\appendix/ command of the standard \LaTeXe\ is a declaration
that changes the way sectional units are numbered. 
But \verb/\appendix/ of \ClassFile\ generates the heading ``Appendix''
and appendix sections are numbered ``1'', ``2'', etc., 
appendix equation numbers are numbered ``(A1)'', ``(A2)'', etc., 
appendix figure numbers are numbered 
``app.\,Fig.\,1'', ``app.\,Fig.\,2'', etc. 

\item
The \verb/\acknowledgment/ command is available if you might 
express your gratitude. 

\item
Authors' profile on page \pageref{profile} is generated with: 
\begin{verbatim}
\begin{biography}
\profile{m}{Denshi Taro}{%
was born in Kumamoto, Japan, 
on August 15, 1972. 
... }
\profile{n}{Denki Hanako}{%
was born in Okayama, Japan, 
on February 25, 1960. 
... }
\end{biography}
\end{verbatim}

\begin{itemize}
\item
The first argument of the \verb/\profile/ command is specified 
by one letter out of seven letters (m, a, s, l, n, h, S) 
the same as the second argument of \verb/\authorentry/. 

The second and third arguments are filled with an author's name and profile 
respectively. 

\item
If EPS (see p.\pageref{EPS}) files of pictures 
of the authors' faces are provided, 
put the EPS files named \texttt{a1.eps}, \texttt{a2.eps}, etc., 
which are followed by the order of authors, 
on the current directory of your computer
(a1.pdf when compiling with \verb/\usepackage[pdftex]{graphicx}/). 
The \verb/\profile/ command automatically reads their files 
and puts their pictures on the left margin. 

The ratio of EPS file must be width : height = 22 : 28, 
because EPS files are read by the following command. 
\begin{verbatim}
\resizebox{22mm}{28mm}
 {\includegraphics{a1.eps}}
\end{verbatim}

If their files don't exist in the current directory, 
simple frames will be generated (see p.\pageref{profile}). 
\end{itemize}

Pictures of the authors' faces may be omitted by 
using the \verb/\profile*/ command instead of the \verb/\profile/ command. 
\end{itemize}

\subsection{Type of the Letter}

\texttt{letter} option must be added as follows. 
\begin{verbatim}
\documentclass[english,letter]{ieej-e}
\end{verbatim}

An abstract should be about 100 words. 
Authors' profiles may be omitted. 

\section{Special Feature of \ClassFile}
\label{generalnote}

\subsection{Formula}

\ClassFile\ includes \texttt{dotseqn.sty} in order to fill 
a space between a formula and a formula number with dots. 

A displayed formula is aligned on the left, a fixed distance (6.5\,mm) 
from the left margin, instead of being centered. 
A formula number is put on the right side, 3.25\,mm from the right margin, 
in the \texttt{equation} and \texttt{eqnarray} environments. 

The following is an example of a displayed formula. 
If you type below, 
\begin{verbatim}
\begin{eqnarray}
\lefteqn{ \int\!\!\!\int_S 
 \left(\frac{\partial V}{\partial x}
 -\frac{\partial U}{\partial y}\right)
 dxdy} \quad\nonumber\\
 &=& \oint_C \left(U \frac{dx}{ds}
      + V \frac{dy}{ds} \right)ds
\end{eqnarray}%
\end{verbatim}
then, you get the following. 
\begin{eqnarray}
\lefteqn{ \int\!\!\!\int_S 
 \left(\frac{\partial V}{\partial x}
 -\frac{\partial U}{\partial y}\right)
 dxdy} \quad\nonumber\\
 &=& \oint_C \left(U \frac{dx}{ds}
      + V \frac{dy}{ds} \right)ds
\end{eqnarray}%

A width of one column is narrow to compose displayed formulas.
Therefore, you should compose equations with the proper length, 
paying attention to the message ``\verb/Overfull/ \verb/\hbox/''.

\subsection{Theorem-like Environment}

If you use the \verb/\newtheorem/ environment, 
pay attention to the following points. 
There are no additional vertical spaces before and after the environment, 
and the text within the environment does not appear in italics. 

An example is given as follows. 
\begin{verbatim}
\newtheorem{theorem}{Theorem}
\begin{theorem}
There are no positive integers such that 
$x^n + y^n = z^n$ for $n>2$. 
I've found a remarkable proof of this fact, 
but there is not enough space 
in the margin [of the book] to write it. 
(Fermat's last theorem). 
\end{theorem}
\end{verbatim}

\newtheorem{theorem}{Theorem}
\begin{theorem}
There are no positive integers such that 
$x^n + y^n = z^n$ for $n>2$. 
I've found a remarkable proof of this fact, 
but there is not enough space 
in the margin [of the book] to write it. 
(Fermat's last theorem). 
\end{theorem}

\subsection{Footnotes}

The footnote begins with ``$^\dagger$'' 
(see p.\pageref{fnsample}). 
As the footnote counter increases, the footnote marks proceed 
as ``$^\dagger$'', ``$^{\dagger\dagger}$'', ``$^{\dagger\dagger\dagger}$''. 
The footnote mark is set to reset at each page.

\subsection{Figures and Tables}

The font size inside the \texttt{figure} and \texttt{table} environments
is set \verb/\footnotesize/ (7\,pt). 

The \texttt{[h]} option, 
one of the arguments of floating environment specifying 
a location where the float may be placed, is not recommended. 
Figures and tables should be located at the top or bottom of a page
by using \texttt{[tb]} etc.\ for the transactions of IEEJ. 

\subsubsection{Including Graphics}
\label{EPS}

Although there are many ways to include pictures and figures in \LaTeX, 
the Encapsulated \PS\ format (EPS) is recommended. 

Here is a simple explanation to insert graphics into the text. 

The \texttt{graphics} or \texttt{graphicx} package must be loaded. 
\texttt{dvips} is one device driver's option 
and it might be changed according to a device driver you use
or might be omitted. 
\begin{verbatim}
\usepackage[dvips]{graphicx}
\end{verbatim}

A graphics file (EPS file) produced by another program can be included 
with the \verb/\includegraphics/ command. 
\begin{verbatim}
\begin{figure}[tb]
\begin{center}
\includegraphics{file.eps}
\end{center}
\caption{...}
\label{fig:1}
\end{figure}
\end{verbatim}

If the option \texttt{scale=0.5} is given, 
the graphics will be scaled by half. 
\begin{verbatim}
\includegraphics[scale=0.5]{file.eps}
\end{verbatim}
You can get the same result as above by using the \verb/\scalebox/ command. 
\begin{verbatim}
\scalebox{0.5}{\includegraphics{file.eps}}
\end{verbatim}

If the option \texttt{width=30mm} is given, 
the width of graphics will be 30\,mm
(with the height proportionally scaled). 
\begin{verbatim}
\includegraphics[width=30mm]{file.eps}
\end{verbatim}
The next is another example using \verb/\resizebox/. 
\begin{verbatim}
\resizebox{30mm}{!}
 {\includegraphics{file.eps}}
\end{verbatim}

Both dimension of width and height can be specified as follows. 
\begin{verbatim}
\includegraphics[width=30mm,height=40mm]
 {file.eps}
\end{verbatim}
or 
\begin{verbatim}
\resizebox{30mm}{40mm}
 {\includegraphics{file.eps}}
\end{verbatim}

For further information about the graphics package, 
see reference book \Cite{latexbook,FMi2}. 


\begin{figure}[t]% fig.1
\setbox0\vbox{%
\hbox{\verb/\begin{figure}[tbp]/}
\hbox{\verb/... floating materials .../}
\hbox{\verb/\capwidth=50mm/}
\hbox{\verb/\caption{An example of caption/}
\hbox{\verb/in English.}/}
\hbox{\verb/\label{fig:1}/}
\hbox{\verb/\end{figure}/}
}
\begin{center}
\fbox{\box0}
\end{center}
\caption{An example of caption in English.}
\label{fig:1}
\end{figure}


\subsubsection{Captions of Floating Environment}
\label{caption}

The width of caption is set 72\,mm (single column) 
and 0.8\verb/\textwidth/ (double column). 
It can be set by changing the value of \verb/\capwidth/ 
(see Fig.\,\ref{fig:1})


\begin{table}[t]% Table 1
\caption{An example of table caption in English.}
\label{table:1}
\setbox0\vbox{%
\hbox{\verb/\begin{table}[b]%[tbp]/}
\hbox{\verb/\caption{An example of table caption in English.}/}
\hbox{\verb/\label{table:1}/}
\hbox{\verb/\begin{center}/}
\hbox{\verb/\begin{tabular}{c|c|c}/}
\hbox{\verb/\hline/}
\hbox{\verb/A & B & C\\/}
\hbox{\verb/\hline/}
\hbox{\verb/X & Y & Z\\/}
\hbox{\verb/\hline/}
\hbox{\verb/\end{tabular}/}
\hbox{\verb/\end{center}/}
\hbox{\verb/\end{table}/}
}
\begin{center}
\begin{tabular}{c|c|c}
\hline
A & B & C\\
\hline
X & Y & Z\\
\hline
\end{tabular}
\halflineskip
\fbox{\box0}
\end{center}
\vskip-2mm
\end{table}


\subsection{Verbatim Environment}

You can change the values of the parameters in the verbatim environment
which is customized for \ClassFile. 
The default settings are: 
\begin{verbatim}
\verbatimleftmargin=0pt
\def\verbatimsize{\normalsize}
\verbatimbaselineskip=\baselineskip
\end{verbatim}
The leftmargin of the verbatim environment is set 0pt. 
The font size is set \verb/\normalsize/. 
The baselineskip is set the same of normal texts. 

For example, those parameters can be changed as follows: 
\begin{verbatim}
\verbatimleftmargin=6.5mm
\def\verbatimsize{\footnotesize}
\verbatimbaselineskip=3mm
\end{verbatim}

\subsection{Bibliography and Citations}

\ClassFile\ includes the \texttt{citesort.sty} 
with a little customized. 
The \texttt{citesort.sty} collapses a list of three or more 
consecutive numbers into a range, and 
sorts the numbers before collapsing them. 
For example, 
``\cite{Bech} \cite{Gr} \cite{PA} \cite{Seroul}
\cite{texbook} \cite{Salomon}''
is transformed into ``\cite{Bech,Gr,PA,Seroul,texbook,Salomon}''. 

The \verb/\cite/ command displays citations as superscript numbers. 
To get ``ref.~\Cite{texbook}'', the \verb/\Cite/ command can be used. 

In the \texttt{thebibliography} environment, 
place references in the right order according to the IEEJ editing style; 
e.g., authors' names, initials, title of article, journal abbreviation, 
volume number, pages, and publication year. 
Journals are italicized as \verb/\itshape/, 
and titles of papers enclosed with ``\ '' in plain text. 

\subsection{Make All Text Pages the Same Height}
\mbox{}\par
\verb/\flushbottom/ is declared in \ClassFile. 
It makes all text pages the same height, 
adding extra vertical space when neccessary to fill out the page. 

\subsection{AMS Packages}

The \AmSLaTeX\ packages are provided to typeset complex equations 
or other mathematical constructions. 
If you would like to use them, 
the \texttt{amsmath} package should be loaded 
with the \texttt{fleqn} option. 
\begin{verbatim}
\usepackage[fleqn]{amsmath}
\end{verbatim}

When the \texttt{amsmath} package is loaded, 
many environments, for example the \texttt{equation}, 
\texttt{align}, \texttt{gather}, 
\texttt{multiline} and \texttt{split} environments etc.,\ 
will not automatically generate dots between a formula and a formula number. 
One primitive way to resolve this problem is 
to typeset as follows at the end of the formula. 
\begin{verbatim}
\rlap{\hbox to 10mm{\ \EqnDots}} 
\end{verbatim}
This puts the leaders of width 10\,mm into a box of width zero, 
extending to the right of the current position. 

While the \texttt{amsmath} package presents many functions, 
only the \verb/\boldsymbol/ command which is 
to be used for individual bold math symbols and bold Greek letters 
is needed, only the \texttt{amsbsy} package should be loaded. 
\begin{verbatim}
\usepackage{amsbsy}
\end{verbatim}

When the \texttt{amssymb} package is loaded, 
many extra math symbols of the \AmSLaTeX\ fonts 
will become available\cite{FMi1}. 
\begin{verbatim}
\usepackage[psamsfonts]{amssymb}
\end{verbatim}

\subsection{Miscellaneous}

\subsubsection{Macros Defined by \ClassFile}
\label{sec:macros}

\begin{itemize}
\item
\verb/\QED/: 
Produces ``$\Box$'' in the end of the proof and so on. 
You would get the same output by using \verb/\hfill $\Box$/. 
But if the end of a paragraph goes to the right margin, 
the character $\Box$ is positioned at the start of a line.
Using \verb/\QED/ will prevent such cases.

\item
\verb/\halflineskip/ and \verb/\onelineskip/: 
Produce a vertical space, 1/2\verb/\baselineskip/, 
1\verb/\baselineskip/ respectively.

\item
As shown in Table \ref{table:2}, 
the macros \verb/\RN/ and \verb/\FRAC/ 
are defined\cite{texbook}. 


\begin{table}[t]% Table 2
\caption{\texttt{\symbol{"5C% "
}}\texttt{FRAC} and \texttt{\symbol{"5C% "
}}\texttt{RN}}
\label{table:2}
\begin{center}
\begin{tabular}{c|c}
\noalign{\hrule height 0.4mm}
\verb/\RN{2}/ & \RN{2} \\
\verb/\RN{117}/ & \RN{117} \\
\hline
\verb/\FRAC{$\pi$}{2}/ & \FRAC{$\pi$}{2}\\
\verb/\FRAC{1}{4}/ & \FRAC{1}{4} \\
\noalign{\hrule height 0.4mm}
\end{tabular}
\end{center}
\end{table}


\item
\verb/\ddash/: 
Produce double ``---''. 
double ``\texttt{---}'' sometimes produce thin space between two ``---''. 
\verb/\ddash/ will prevent such a case. 
\end{itemize}

\section{Typographic Notes}
\label{typesetting}

\subsection{How to Prevent Typographic Errors}

\begin{enumerate}
\item
You should pay attention to a space after a period. 

``\TeX\ simply assumes that a period ends a sentence unless
it follows an uppercase letter. This works most of the time, 
but not always---abbreviations like `etc.'\ being the most
common exception. You tell \TeX\ that a period doesn't 
end a sentence by using a \verb*/\ / command (a \verb/\/
character followed by a space or the end of a line) 
to make the space after the period.''

``On the rare occasions that a sentence-ending period
follows an uppercase letter, you will have to tell \TeX\ 
that the period ends the sentence. You do this by preceding 
the period with a \verb/\@/ command.''\cite{latexbook}

\texttt{Beans} \texttt{(lima, etc.)}\verb/\/ 
\texttt{have vitamin B}\verb/\@/. 

\item
``Line breaking should be prevented at certain interword spaces.
 ... Trying \~\ (a tilde character) produces an ordinary interword
space at which \TeX\ will never break a line.''\cite{latexbook}

\verb/Mr.~Jones/, 
\verb/Figure~\ref{fig:1}/, 
\verb/(1)~gnats/.

\item
With respect to Figure, Section, Equation, when these words appear at 
the beginning of a sentence, they should be spelt out in full, 
e.g., ``Figure~1 shows...'' is used. 
When they appear in the middle of a sentence, 
abbreviations, e.g.,\ ``in Fig.\,1'', ``in Sect.\,2'', 
``in Eq.\,3'' should be used.

\item
There should be no space after opening or before closing parentheses, 
as in \verb*/( word )/.

\item
There are many cases of an inappropriate application of a \verb/\\/ 
or \verb/\newline/ command except in the tabular environment etc., 
such as two \verb/\\/ commands in succession 
or \verb/\\/ command just before a blank line. 
They will often cause warning messages like 
\texttt{Underfull} \verb/\hbox .../, 
as a result it would often prevent you 
from finding important warning messages. 
The use of \verb/\par\noindent/ or \verb/\hfil\break/ commands 
is recommended and gives you the same effect without warning messages. 

\item
There are some cases of an inappropriate application of a \verb/\\/
in descriptions such as program lists.
Use of the \texttt{tabbing} environment or \texttt{list} environment 
is recommended. 

\item
The difference in the use of the hyphen (\texttt{-}), 
en dash (\texttt{--}) and 
em dash (\texttt{---}) should be noted. 
A hyphen is used in connecting English-language words 
such as `well-known', and an en dash is used 
when expressing a range such as `pp.298--301'.
The em dash is even longer---it's used as punctuation. 

\item
The difference when hyphen and en dash are used in maths mode 
should also be noted. 
Some examples are given below. 
\begin{verbatim}
$A^{\mathrm{b}\mbox{-}\mathrm{c}}$
\end{verbatim}
$A^{\mathrm{b}\mbox{-}\mathrm{c}}$ $\Rightarrow$ hyphen
\begin{verbatim}
$A^{\mathrm{b}\mbox{--}\mathrm{c}}$
\end{verbatim}
$A^{\mathrm{b}\mbox{--}\mathrm{c}}$ $\Rightarrow$ en dash
\begin{verbatim}
$A^{b-c}$
\end{verbatim}
$A^{b-c}$ $\Rightarrow$ minus sign\par

\item
The less-than sign ``$<$'' (\verb/</, a relation) should not be confused 
with ``$\langle$'' (\verb/\langle/, a delimiter). 
The same is true for the greater-than sign ``$>$'' and ``$\rangle$''.

\item
A unary operator and a binary operator: 
``A $+$ or $-$ that begins a formula (or certain subformulas) is 
assumed to be a unary operator, so typing \verb/$-x$/ produces 
$-x$ and typing \verb/$\sum - x_{i}$/ produces $\sum - x_{i}$, 
with no space between the ``$-$'' and ``x''. 
If the formula is part of a larger one that is being split 
across lines, \TeX\ must be told that the $+$ or $-$ is 
a binary operator. This is done by starting the formula with an
invisible first term, produced by an \verb/\mbox/ command
with a null argument.''\cite{latexbook} 

\begin{verbatim}
\begin{eqnarray}
y &=& a + b + c + ... + e\\
  & & \mbox{} + f + ... 
\end{eqnarray}
\end{verbatim}

\item
\verb/\allowbreak/ may be used within long maths formulas in paragraphs 
instead of using \verb/\hfil\break/, \verb/\\/ or \verb/\linebreak/, 
since \TeX\ is reluctant to break lines there. 
\end{enumerate}

\subsection{How to Handle Long Formulas}
\label{longformula}

Here are some explanations how to handle long formulas, 
for example, overhanged equations, equations overriding the equation number, 
and so forth. 

\halflineskip
\hfuzz10pt

\noindent
\textbf{Example 1}: 
\begin{equation}
 y=a+b+c+d+e+f+g+h+i+j+k+l+m
\end{equation}
The equation is too long, and the space between the equation 
and the equation number are too narrow and sometimes the equation number 
moves to the right. 
In this case the \verb/\!/ command is useful. 

``The \verb/\!/ acts like a backspace, removing the same space amount 
of space that \verb/\,/ adds.''\cite{latexbook}
\begin{verbatim}
\begin{equation}
 y\!=\!a\!+\!b\!+\!c\!+\! ... \!+\!m
\end{equation}
\end{verbatim}
\begin{equation}
 y\!=\!a\!+\!b\!+\!c\!+\!d\!+\!e\!+\!f
     \!+\!g\!+\!h\!+\!i\!+\!j\!+\!k\!+\!l\!+\!m
\end{equation}

\halflineskip

\noindent
\textbf{Example 2}: 
Using \texttt{eqnarray} environment 
instead of \texttt{equation} environment. 
\begin{verbatim}
\begin{eqnarray}
 y &=& a+b+c+d+e+f+g+h\nonumber\\
   & & \mbox{}+i+j+k+l+m+n+o
\end{eqnarray}
\end{verbatim}
To typeset above, you will get the following output. 
\begin{eqnarray}
 y &=& a+b+c+d+e+f+g+h\nonumber\\
   & & \mbox{}+i+j+k+l+m+n+o
\end{eqnarray}

\halflineskip

\noindent
\textbf{Example 3}: 
To change the value of \verb/\mathindent/ is 
to change the position that the equation begins%
\footnote{This explanation is appropriate to left-aligns displayed formulas, 
not to centering formulas.}. 
\label{fnsample}%
\begin{verbatim}
\mathindent=0mm % <-- [1]
\begin{equation}
 y=a+b+c+d+e+f+g+h+i+j+k+l+m
\end{equation}
\mathindent=6.5mm % <-- [2] default value
\end{verbatim}
To typeset above (see \texttt{[1]}), you will get the following output. 
\mathindent=0mm
\begin{equation}
 y=a+b+c+d+e+f+g+h+i+j+k+l+m
\end{equation}
\mathindent=6.5mm

The value of \verb/\mathindent/ must be restored (see \texttt{[2]}). 

\halflineskip

\noindent
\textbf{Example 4}: 
\begin{equation}
 \int\!\!\!\int_S \left( \frac{\partial V}{\partial x}
 - \frac{\partial U}{\partial y} \right)dxdy=
 \oint_C \left(U\frac{dx}{ds}+V\frac{dy}{ds}\right)ds
 \hskip-3.5mm %% --> oppress `Overful \hbox ...' message 
\end{equation}

The equation is too long and overrides the equation number. 
In this case the \verb/\lefteqn/ command is useful. 
It can be used for splitting long formulas across lines as follows. 
\begin{verbatim}
\begin{eqnarray}
 \lefteqn{
  \int\!\!\!\int_S 
  \left(\frac{\partial V}{\partial x}
  -\frac{\partial U}{\partial y}\right) dxdy
 }\quad \nonumber\\
 &=& \oint_C \left(U \frac{dx}{ds}
      + V \frac{dy}{ds} \right)ds
\end{eqnarray}
\end{verbatim}
To typeset above, you will get the following output. 
\begin{eqnarray}
\lefteqn{
 \int\!\!\!\int_S 
  \left(\frac{\partial V}{\partial x}
   -\frac{\partial U}{\partial y}\right)
    dxdy
 }\quad\nonumber\\
 &=& \oint_C \left(U \frac{dx}{ds}
      + V \frac{dy}{ds} \right)ds
\end{eqnarray}

\halflineskip

\noindent
\textbf{Example 5}: 
A matrix using the \texttt{array} environment. 
\begin{equation}
A = \left(
  \begin{array}{cccc}
   a_{11} & a_{12} & \ldots & a_{1n} \\
   a_{21} & a_{22} & \ldots & a_{2n} \\
   \vdots & \vdots & \ddots & \vdots \\
   a_{m1} & a_{m2} & \ldots & a_{mn} \\
  \end{array}
    \right) \label{eq:ex1}
\end{equation}

The width of a matrix can be shrunk
by changing the value of \verb/\arraycolsep/ or 
using an \texttt{@}-expression (\verb/@{}/). 
\begin{verbatim}
\begin{equation}
\arraycolsep=3pt %     <--- [1]
A = \left(
  \begin{array}{@{\hskip2pt}%%  <-- [2]
                cccc
                @{\hskip2pt}%%  <-- [2]
               }
   a_{11} & a_{12} & \ldots & a_{1n} \\
   a_{21} & a_{22} & \ldots & a_{2n} \\
   \vdots & \vdots & \ddots & \vdots \\
   a_{m1} & a_{m2} & \ldots & a_{mn} \\
  \end{array}
    \right) 
\end{equation}
\end{verbatim}

The \verb/\arraycolsep/ dimension is half the width of a horizontal space 
between columns in the \texttt{array} environment. 
A matrix using the \texttt{array} environment can be shrunk
by changing the value of \verb/\arraycolsep/ (see \texttt{[1]}). 
And also it can be shrunk by using \texttt{@}-expression 
(see \texttt{[2]}). 

\begin{equation}
\arraycolsep=3pt
A = \left(
  \begin{array}{@{\hskip2pt}cccc@{\hskip2pt}}
   a_{11} & a_{12} & \ldots & a_{1n} \\
   a_{21} & a_{22} & \ldots & a_{2n} \\
   \vdots & \vdots & \ddots & \vdots \\
   a_{m1} & a_{m2} & \ldots & a_{mn} \\
  \end{array}
    \right) \label{eq:ex2}
\end{equation}
Compare Eqs.\,(\ref{eq:ex1}) and (\ref{eq:ex2}). 

\halflineskip
\hfuzz.5pt

\noindent
\textbf{Example 6}: A matrix using a \verb/\pmatrix/. 
\begin{verbatim}
\begin{equation}
 \def\quad{\hskip.75em\relax}% <-- [1]
 %% default setting is \hskip1em
 A = \pmatrix{
      a_{11} & a_{12} & \ldots & a_{1n} \cr
      a_{21} & a_{22} & \ldots & a_{2n} \cr
      \vdots & \vdots & \ddots & \vdots \cr
      a_{m1} & a_{m2} & \ldots & a_{mn} \cr
     }
\end{equation}
\end{verbatim}
In the case of the equation using \verb/\pmatrix/, 
the definition of \verb/\quad/ can be changed (see \texttt{[1]}). 

If \texttt{amsmath} packages is loaded, 
the \texttt{pmatrix} environment 
(\verb/\begin{pmatrix}/ and \verb/\end{pmatrix}/) must be selected 
instead of \verb/\pmatrix/.
In that case the explanation on Example 5 is useful. 

%%%
%\[
% \begin{pmatrix}
%      a_{11} & a_{12} & \ldots & a_{1n} \cr
%      a_{21} & a_{22} & \ldots & a_{2n} \cr
%      \vdots & \vdots & \ddots & \vdots \cr
%      a_{m1} & a_{m2} & \ldots & a_{mn} \cr
% \end{pmatrix}
%\]
%\[
% \arraycolsep1pt
% \begin{pmatrix}
%      a_{11} & a_{12} & \ldots & a_{1n} \cr
%      a_{21} & a_{22} & \ldots & a_{2n} \cr
%      \vdots & \vdots & \ddots & \vdots \cr
%      a_{m1} & a_{m2} & \ldots & a_{mn} \cr
% \end{pmatrix}
%\]
%%%

\begin{thebibliography}{99}
\bibitem{texbook}
D.E. Knuth, The \TeX{}book, Addison-Wesley (1994)

\bibitem{Seroul}
R. Seroul \& S. Levy: 
A Beginner's Book of \TeX, Springer-Verlag (1989)

\bibitem{Salomon}
D. Salomon: 
The Advanced \TeX{}book, 
Springer-Verlag (1995)

\bibitem{Eijkhout}
V. Eijkhout: \TeX\ by Topic, Addison-Wesley (1991)

\bibitem{PA}
P.W. Abrahams: \TeX\ for the Impatient, Addison-Wesley (1992)

\bibitem{Bech}
S. von Bechtolsheim: \TeX\ in Practice, Springer-Verlag (1993)

%\bibitem{Kopka}
%H. Kopka \& P.W. Daly: 
%A Guide to \LaTeX, Addison-Wesley (1993)

\bibitem{Gr}
G. Gr\"{a}tzer: Math into \TeX--A Simple Introduction to \AmSLaTeX,
Birkh\"{a}user (1993)

\bibitem{Walsh}
N. Walsh: Making \TeX\ Work, O'Reilly \& Associates (1994)%%\hfil\break
%%http://makingtexwork.sourceforge.net/mtw/

\bibitem{latexbook}
L. Lamport, \LaTeX: A Document Preparation System, Second Edition,  
Addison-Wesley (1994) 

\bibitem{FMi1}
M. Goossens, F. Mittelbach \& A. Samarin: 
The \LaTeX\ Companion, Addison-Wesley (1994)

\bibitem{FMi2}
M. Goossens, S. Rahts, and F. Mittelbach: 
The \LaTeX\ Graphics Companion, Addison-Wesley (1997)

%\bibitem{FMi3}
%M. Goossens, and S. Rahts: 
%The \LaTeX\ Web Companion,  
%Addison-Wesley (1999)

%\bibitem{Lipkin}
%B.S. Lipkin: 
%\LaTeX\ for Linux, Springer-Verlag New York (1999)
\end{thebibliography}

\raggedbottom
\appendix

\section{Printing on A4 paper and making pdf file}
\label{sec:pdf}

\begin{itemize}
\item
If you print a manuscript on A4 paper by using 
dvips printer driver, the following parameter might be set. 
\begin{verbatim}
dvips -Pprinter -t a4 -O 0in,0in file.dvi
\end{verbatim}
\texttt{printer} is a name of printer. 
``\verb/-t a4 -O 0in,0in/'' option might be omitted. 

\item
You can directly make a pdf file by using pdflatex, 
or convert a dvi file to a pdf file by using dvips and Acrobat Distiller 
or dvipdfmx. 

\item
If you convert a dvi file to a pdf file, 
you must first convert a dvi file to a ps file 
(\texttt{printer} is your printer name): 
\begin{verbatim}
dvips -Pprinter -t a4 -O 0in,0in
 -o file.ps file.dvi
\end{verbatim}
``\verb/-t a4 -O 0in,0in/'' option might be omitted. 
Then, convert a ps file to pdf file by using Acrobat Distiller. 

Otherwise, you may convert a dvi file to a pdf file by using dvipdfmx. 
\begin{verbatim}
dvipdfmx -p a4 -x 1in -y 1in
 -o file.pdf file.dvi
\end{verbatim}
``\verb/-p a4 -x 1in -y 1in/'' option might be omitted. 
\end{itemize}

\section{Omitted Commands}

Some commands which is not required by \ClassFile\ are omitted. 
These commands are 
\verb/\tableofcontents/, 
\verb/\titlepage/, 
\verb/\part/, 
\verb/\theindex/, 
\texttt{headings}, 
\texttt{myheadings\/} 
and the related commands. 


\end{document}
